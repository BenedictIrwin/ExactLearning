\documentclass[journal=jcisd8,manuscript=article,layout=onecolumn,pdftex,floatfix,amsmath,amssymb,10pt]{achemso}
\usepackage{hyperref,url,color,upgreek,amssymb,amsmath}
\usepackage{graphicx}
\SectionNumbersOn
\AbstractOn

\title{Notes on TSNE}
\author{Benedict W. J.~Irwin}
\email{ben.irwin@optibrium.com}
\affiliation{Optibrium, F5-6 Blenheim House, Cambridge Innovation Park,
Denny End Road, Cambridge, CB25 9PB, United Kingdom}

\begin{document}
\begin{abstract}
Integral Schrödinger Equation
\end{abstract}

\section{Introduction}

$$
\psi(\mathbf{r}) = \frac{m}{2\pi \hbar^2} \int \frac{e^{ik|r-r_0|}}{|r-r_0|} V(r_0)\psi(r_0)d^3r_0
$$
$$
\psi(\mathbf{r}) = \sum_{j=0}^\infty \frac{(ik)^j}{j!} \frac{m}{2\pi \hbar^2} \int |r-r_0|^{j-1} V(r_0)\psi(r_0)d^3r_0
$$

We can restrict to potentials which are infinite everywhere except the positive quadrant.
$$
\psi(x,y,z) =  \frac{m}{2\pi \hbar^2} \sum_{j=0}^\infty \frac{(ik)^j}{j!} \int_0^\infty \int_0^\infty \int_0^\infty \sqrt{(x-x_0)^2+(y-y_0)^2+(z-z_0)^2}^{j-1} V(x_0,y_0,z_0)\psi(x_0,y_0,z_0)dx_0 dy_0 dz_0
$$

Here we see a very funky alternative to the multi dimensional Mellin-transform.
In 1-D they might be equivalent.

Through this new "Mellin transform", the wavefunction is function, which is the sum of the Mellin transform of the potential times itself.

The integral looks like the expectation of the distance between the point $(x,y,z)$ and the point $(x_0,y_0,z_0)$, raised to some power.

If we expected similar behaviour to the Ramanujan Master theorem in this case, for the potential multipled by the wavefunction, we could posit
$$
V(x_0,y_0,z_0)\psi(x_0,y_0,z_0) = \sum_{\mathbf{k}=0}^\infty \chi(\mathbf{k})\Phi_{V\psi}(\mathbf{k})\mathbf{r}^{\mathbf{Ak+b}}
$$
We might then posit that
$$
\int_0^\infty \sqrt{(x-x_0)^2+(y-y_0)^2+(z-z_0)^2}^{j-1} V(x_0,y_0,z_0)\psi(x_0,y_0,z_0)\mathcal{D}\mathbf{r} \approx \frac{\Phi_{V\psi}(\mathbf{k}^*)}{\mathrm{det(\mathbf{A})}}\prod_{i=1}^3 \Gamma(-k_i^*)
$$
for the solution $\mathbf{Ak^{*}+b} = \mathbf{j}$, then the wave-function itself is of the form 
\begin{equation}
\frac{m}{2\pi \hbar^2} \sum_{j=0}^\infty \frac{(ik)^j}{j\Gamma(j+1)} \frac{\Phi_{V\psi}(\mathbf{A}^{-1}(\mathbf{j-b}))}{\mathrm{det(\mathbf{A})}}\prod_{i=1}^3 \Gamma(-k_i^*)
\end{equation}

Of course, for spherically symmetric potentials:
$$
\psi(r) =  \frac{m}{2\pi \hbar^2} \sum_{j=0}^\infty \frac{(ik)^j}{j!} \int_0^\infty \int_0^\pi \int_0^{2\pi} |r-r_0|^{j-1} V(r_0)\psi(r_0) \sin \theta r_0^2 d\phi d\theta dr_0
$$

$$
\int_0^\infty \sqrt{(x-x_0)^2+(y-y_0)^2+(z-z_0)^2}^{j-1} Q(x_0,y_0,z_0)\mathcal{D}\mathbf{r} \approx \frac{\Phi_{V\psi}(\mathbf{k}^*)}{\mathrm{det(\mathbf{A})}}\prod_{i=1}^3 \Gamma(-k_i^*)
$$

Essentially interested in the transform


\section{Reverse Engineering QM}
What we know from real wavefunctions.

From Griffiths


$$
\frac{1}{\sqrt{\pi a^3}} e^{-r/a} = - \frac{m}{2\pi \hbar^2}\int \frac{e^{-|r-r_0|/a}}{|r-r_0|}(\frac{-\hbar^2}{m a r})\frac{1}{\sqrt{\pi a^3}} e^{-r/a} d\tau
$$

$$
 e^{-r/a} = \frac{1}{2\pi }\int \frac{e^{-|r-r_0|/a}}{|r-r_0|}(\frac{1}{a r}) e^{-r/a} d\tau
$$

We expand into our bracket symbol
$$
 e^{-r/a} = \sum_{k=0}^\infty \frac{(-1)^k}{k!} \frac{1}{2\pi }\int \left(\frac{|r-r_0|}{a}\right)^{k-1}\frac{1}{r} e^{-r/a} d\tau
$$

$$
 e^{-r/a} = \sum_{k=0}^\infty \frac{(-1)^k}{k!} \frac{1}{2\pi }\int \left(\frac{|r-r_0|}{a}\right)^{k-1}\frac{e^{-r/a}}{r}  d\tau
$$
then 


So we have 

$$
\int_0^\infty \int_0^\infty \sqrt{(x-x_0)^2+(y-y_0)^2}^{j-1} dx dy
$$
this can be assigned the series

$$
\int_0^\infty \int_0^\infty \sum_{n_1 \ge 0} \sum_{n_2 \ge 0} \phi_{n_1,n_2}(x-x_0)^{2n_1}(y-y_0)^{2n_2}\frac{1}{\Gamma(\frac{1-j}{2})} \langle \frac{1-j}{2} + n_1 + n_2 \rangle dx dy
$$

then we again apply the multinomial power rule
$$
\int_0^\infty \int_0^\infty \sum_{t_1\ge 0} \sum_{t_2 \ge 0}\sum_{s_1 \ge 0} \sum_{s_2 \ge 0} \sum_{n_1 \ge 0} \sum_{n_2 \ge 0} \phi_{s_1,s_2} \phi_{t_1,t_2} \phi_{n_1,n_2} \frac{\langle -2n_1 + s_1 + s_2 \rangle \langle -2n_2 + t_1 + t_2 \rangle}{\Gamma(-2n_1)\Gamma(-2n_2)}  x^{s_1} (-x_0)^{s_2} y^{t_1} (-y_0)^{t_2} \frac{\langle \frac{1-j}{2} + n_1 + n_2 \rangle}{\Gamma(\frac{1-j}{2})}  dx dy
$$

The integrals also evaluate to terms in the bracket series
$$
\sum_{\mathbf{t}\ge 0}\sum_{\mathbf{s} \ge 0} \sum_{\mathbf{n} \ge 0}  \phi_{s_1,s_2} \phi_{t_1,t_2} \phi_{n_1,n_2} \frac{\langle -2n_1 + s_1 + s_2 \rangle \langle -2n_2 + t_1 + t_2 \rangle}{\Gamma(-2n_1)\Gamma(-2n_2)\Gamma(\frac{1-j}{2})} \langle s_1 + 1\rangle \langle t_1 + 1\rangle  \langle \frac{1-j}{2} + n_1 + n_2 \rangle (-x_0)^{s_2}  (-y_0)^{t_2}
$$
Then we have 5 brackets and 6 sums so we will requite an additional series in terms of the free parameters?

We know that we can write this series in terms of the solutions to the equations
\begin{equation}
\begin{bmatrix}
-2 & 0 & 1 & 1 & 0 & 0 \\
0 & -2 & 0 & 0 & 1 & 1 \\
0 & 0 & 1 & 0 & 0 & 0 \\
0 & 0 & 0 & 0 & 1 & 0 \\
1 & 1 & 0 & 0 & 0 & 0 \\
0 & 0 & 0 & 0 & 0 & 0 \\
\end{bmatrix}
\begin{bmatrix}
n_1 \\ n_2 \\ s_1 \\ s_2 \\ t_1 \\ t_2
\end{bmatrix}
=
\begin{bmatrix}
0 \\ 0 \\ -1 \\ -1 \\ (j-1)/2 \\ ?
\end{bmatrix}
\end{equation}

We need to contribute all maximal rank possibilities and sum them, together.


\section{Moments of Wavefunction}
We can attempt to bring more variables in by closing each side with another triple Mellin transform, this will liberate the moments
$$
\iiint_0^\infty \mathbf{r}^{\mathbf{s-1}}\psi(\mathbf{r})\mathcal{D}\mathbf{r} =  \frac{m}{2\pi \hbar^2} \iiint_0^\infty \mathbf{r}^{\mathbf{s-1}} \sum_{j=0}^\infty \frac{(ik)^j}{j!} \int_0^\infty \int_0^\infty \int_0^\infty \sqrt{(x-x_0)^2+(y-y_0)^2+(z-z_0)^2}^{j-1} V(x_0,y_0,z_0)\psi(x_0,y_0,z_0)\mathcal{D}\mathbf{r_0}\mathcal{D}\mathbf{r}
$$

Then we should find $6$ brackets, and we perform one expansion for the exponential (Green's function), two for the multinomial, three for each of the binomials inside. But then we need another 3 or 6 for the expansion of $V\psi$, or both $V$ and $\psi$.


\end{document}