\documentclass[journal=jcisd8,manuscript=article,layout=onecolumn,pdftex,floatfix,amsmath,amssymb,10pt]{achemso}
\usepackage{hyperref,url,color,upgreek,amssymb,amsmath}
\usepackage{graphicx}
\SectionNumbersOn
\AbstractOn

\title{MGF General Function}
\author{Benedict W. J.~Irwin}
\email{ben@.com}
\affiliation{Cambridge, United Kingdom}

\begin{document}
\begin{abstract}
Bilateral Laplace Transform of Hypergeom
\end{abstract}

\section{Introduction}
Consider defining a new function 
$$
Q(a,b;c;t) = \mathcal{B}_{x\to t}\left[\;_2F_1(a,b;c;-x)\right]
$$
this is a fundamental generalised function, but it does not have a compact form
\begin{align}
\frac{2 i \pi ^2 (-t)^b \Gamma (c) \csc (\pi  (a-b)) \, _1F_1(b-c+1;-a+b+1;t)}{t \Gamma (a) \Gamma (b) \Gamma (-a+b+1) \Gamma (c-b)} \\ - \frac{2 i \pi ^2 (-t)^a \Gamma (c) \csc (\pi  (a-b)) \,
    _1F_1(a-c+1;a-b+1;t)}{t \Gamma (a) \Gamma (b) \Gamma (a-b+1) \Gamma (c-a)} 
\\ \text{ if }\Re(a+b-c)<1\land \Re(t)\leq 0\land (\Re(t)<0\lor (\Re(a)>-2\land \Re(b)>-2))
\end{align}
    
The function represents the moment generating function (MGF), for any distribution that can be represented by the original hypergeometric function.

It is nice to then consider the Mellin transform of the above to get a more regular structure.


We can simplify it slightly by expanding the $\csc$ functions.

For $\;_2F_3$ we have:
\begin{align}
\frac{\pi \Gamma (\text{b1}) \Gamma (\text{b2}) }{t \Gamma (\text{a1}) \Gamma (\text{a2}) \Gamma (\text{a3})}
\\ -\frac{ \Gamma (\text{a1}-\text{a3}) \Gamma (\text{a2}-\text{a3}) \left(-t^{\text{a3}} \csc (\pi  \text{a3})-i (-t)^{\text{a3}}+(-t)^{\text{a3}} \cot (\pi
    \text{a3})\right) \, _2F_2(\text{a3}-\text{b1}+1,\text{a3}-\text{b2}+1;-\text{a1}+\text{a3}+1,-\text{a2}+\text{a3}+1;t)}{ \Gamma (\text{b1}-\text{a3}) \Gamma
    (\text{b2}-\text{a3})}
\\ -\frac{ \Gamma (\text{a1}-\text{a2}) \Gamma (\text{a3}-\text{a2}) \left(-t^{\text{a2}} \csc (\pi  \text{a2})-i (-t)^{\text{a2}}+(-t)^{\text{a2}} \cot (\pi
    \text{a2})\right) \, _2F_2(\text{a2}-\text{b1}+1,\text{a2}-\text{b2}+1;-\text{a1}+\text{a2}+1,\text{a2}-\text{a3}+1;t)}{\Gamma (\text{b1}-\text{a2}) \Gamma
    (\text{b2}-\text{a2})}
\\ -\frac{ \Gamma (\text{a2}-\text{a1}) \Gamma (\text{a3}-\text{a1}) \left(-t^{\text{a1}} \csc (\pi  \text{a1})-i (-t)^{\text{a1}}+(-t)^{\text{a1}} \cot (\pi
    \text{a1})\right) \, _2F_2(\text{a1}-\text{b1}+1,\text{a1}-\text{b2}+1;\text{a1}-\text{a2}+1,\text{a1}-\text{a3}+1;t)}{\Gamma (\text{b1}-\text{a1}) \Gamma
    (\text{b2}-\text{a1})}    
\\ \text{ if }\Re(\text{a1})+\Re(\text{a2})+\Re(\text{a3})-\Re(\text{b1})-\Re(\text{b2})<1
\end{align}
where $\Gamma\Gamma(i,j,k) = \Gamma(i-k)\Gamma(j-k)$, and $C(i,t) = \left(-t^{\text{i}} \csc (\pi  \text{i})-i (-t)^{\text{i}}+(-t)^{\text{i}} \cot (\pi\text{i})\right)$
\begin{align}
\frac{\pi \Gamma (\text{b1}) \Gamma (\text{b2}) }{t \Gamma (\text{a1}) \Gamma (\text{a2}) \Gamma (\text{a3})} \times \bigg(
\\ -\frac{ \Gamma\Gamma(\text{a1},\text{a2},\text{a3}) C(a3,t) \, _2F_2(\text{a3}-\text{b1}+1,\text{a3}-\text{b2}+1;-\text{a1}+\text{a3}+1,-\text{a2}+\text{a3}+1;t)}{ \Gamma\Gamma(\text{b1},\text{b2},\text{a3})}
\\ -\frac{ \Gamma\Gamma(\text{a1},\text{a3},\text{a2}) C(a2,t) \, _2F_2(\text{a2}-\text{b1}+1,\text{a2}-\text{b2}+1;-\text{a1}+\text{a2}+1,\text{a2}-\text{a3}+1;t)}{\Gamma\Gamma(\text{b1},\text{b2},\text{a2})}
\\ -\frac{ \Gamma\Gamma(\text{a2},\text{a3},\text{a1}) C(a1,t) \, _2F_2(\text{a1}-\text{b1}+1,\text{a1}-\text{b2}+1;\text{a1}-\text{a2}+1,\text{a1}-\text{a3}+1;t)}{\Gamma\Gamma(\text{b1},\text{b2},\text{a1})} \bigg)   
\\ \text{ if }\Re(\text{a1})+\Re(\text{a2})+\Re(\text{a3})-\Re(\text{b1})-\Re(\text{b2})<1
\end{align}
define 
\begin{align}
f(i;a1,a2;b1,b2,t) = \, _2F_2(i-\text{b1}+1,i-\text{b2}+1;-\text{a1}+i+1,-\text{a2}+i+1;t) \\
\, _2F_2(\text{a3}-\text{b1}+1,\text{a3}-\text{b2}+1;-\text{a1}+\text{a3}+1,-\text{a2}+\text{a3}+1;t) \\
\, _2F_2(\text{a2}-\text{b1}+1,\text{a2}-\text{b2}+1;-\text{a1}+\text{a2}+1,\text{a2}-\text{a3}+1;t) \\
\, _2F_2(\text{a1}-\text{b1}+1,\text{a1}-\text{b2}+1;\text{a1}-\text{a2}+1,\text{a1}-\text{a3}+1;t)
\end{align}

In general we have 
$$
M_X(t) = \sum_{k=0}^\infty \frac{E(X^k)t^k}{k!}
$$
by Ramanujan Master theorem we have
$$
\mathcal{M}_s[M_X(-t)] = \Gamma(s)E(X^{-s})
$$
for our pdf, we had a fingerprint defined as 
$$
\mathcal{M}[p(x)](s) = \phi(s) = E[X^{s-1}]
$$
$$
\mathcal{M}[p(x)](s+1) = \phi(s+1) = E[X^{s}]
$$
$$
\mathcal{M}[p(x)](1-s) = \phi(1-s) = E[X^{-s}]
$$
so we can say the Mellin transform of the MGF gives
$$
\mathcal{M}_s[M_X(-t)] = \Gamma(s)\phi(1-s)
$$
moreover 
$$
M_X(-t) = \mathcal{M}^{-1}_t [\Gamma(s)\phi(1-s)]
$$
This means for very complicated functions such as Meijer-G and others, where we know the form of the fingerprint we can write the expression for the generalised moment generating function.
This gives $M_X(t)$ for $\;_2F_1(a,b,c,-x)$ as
$$
 \frac{(c-1) \, _2F_2(1,2-c;2-a,2-b;-t)}{(a-1) (b-1)}+\frac{\pi ^2 (-t)^{a-1} \csc (\pi  a) \Gamma (c) \csc (\pi  (b-a)) \, _1F_1(a-c+1;a-b+1;-t)}{\Gamma (a) \Gamma (b) \Gamma (a-b+1) \Gamma (c-a)}+\frac{\pi ^2
    (-t)^{b-1} \csc (\pi  b) \Gamma (c) \csc (\pi  a-\pi  b) \, _1F_1(b-c+1;-a+b+1;-t)}{\Gamma (a) \Gamma (b) \Gamma (-a+b+1) \Gamma (c-b)}
$$
which is more similar to the regular Laplace transform...
    
For the next order it gives
    $$
    \frac{(\text{b1}-1) (\text{b2}-1) \, _3F_3(1,2-\text{b1},2-\text{b2};2-\text{a1},2-\text{a2},2-\text{a3};-t)}{(\text{a1}-1) (\text{a2}-1) (\text{a3}-1)}+\frac{\pi ^3 (-t)^{\text{a1}-1} \csc (\pi  \text{a1}) \Gamma
    (\text{b1}) \Gamma (\text{b2}) \csc (\pi  (\text{a2}-\text{a1})) \csc (\pi  (\text{a3}-\text{a1})) \, _2F_2(\text{a1}-\text{b1}+1,\text{a1}-\text{b2}+1;\text{a1}-\text{a2}+1,\text{a1}-\text{a3}+1;-t)}{\Gamma (\text{a1}) \Gamma
    (\text{a2}) \Gamma (\text{a3}) \Gamma (\text{a1}-\text{a2}+1) \Gamma (\text{a1}-\text{a3}+1) \Gamma (\text{b1}-\text{a1}) \Gamma (\text{b2}-\text{a1})}-\frac{\pi ^3 (-t)^{\text{a2}-1} \csc (\pi  \text{a2}) \Gamma (\text{b1}) \Gamma
    (\text{b2}) \csc (\pi  \text{a1}-\pi  \text{a2}) \csc (\pi  (\text{a2}-\text{a3})) \, _2F_2(\text{a2}-\text{b1}+1,\text{a2}-\text{b2}+1;-\text{a1}+\text{a2}+1,\text{a2}-\text{a3}+1;-t)}{\Gamma (\text{a1}) \Gamma (\text{a2}) \Gamma
    (\text{a3}) \Gamma (-\text{a1}+\text{a2}+1) \Gamma (\text{a2}-\text{a3}+1) \Gamma (\text{b1}-\text{a2}) \Gamma (\text{b2}-\text{a2})}+\frac{\pi ^3 (-t)^{\text{a3}-1} \csc (\pi  \text{a3}) \Gamma (\text{b1}) \Gamma (\text{b2}) \csc
    (\pi  \text{a1}-\pi  \text{a3}) \csc (\pi  \text{a2}-\pi  \text{a3}) \, _2F_2(\text{a3}-\text{b1}+1,\text{a3}-\text{b2}+1;-\text{a1}+\text{a3}+1,-\text{a2}+\text{a3}+1;-t)}{\Gamma (\text{a1}) \Gamma (\text{a2}) \Gamma (\text{a3})
    \Gamma (-\text{a1}+\text{a3}+1) \Gamma (-\text{a2}+\text{a3}+1) \Gamma (\text{b1}-\text{a3}) \Gamma (\text{b2}-\text{a3})}
    $$

For the next order it gives
\begin{align}
 \frac{(\text{b1}-1) (\text{b2}-1) (\text{b3}-1) \, _4F_4(1,2-\text{b1},2-\text{b2},2-\text{b3};2-\text{a1},2-\text{a2},2-\text{a3},2-\text{a4};-t)}{(\text{a1}-1) (\text{a2}-1) (\text{a3}-1) (\text{a4}-1)}
 \\+\frac{\pi ^4
    (-t)^{\text{a1}-1} \csc (\pi  \text{a1}) \Gamma (\text{b1}) \Gamma (\text{b2}) \Gamma (\text{b3}) \csc (\pi  (\text{a2}-\text{a1})) \csc (\pi  (\text{a3}-\text{a1})) \csc (\pi  (\text{a4}-\text{a1})) \,
    _3F_3(\text{a1}-\text{b1}+1,\text{a1}-\text{b2}+1,\text{a1}-\text{b3}+1;\text{a1}-\text{a2}+1,\text{a1}-\text{a3}+1,\text{a1}-\text{a4}+1;-t)}{\Gamma (\text{a1}) \Gamma (\text{a2}) \Gamma (\text{a3}) \Gamma (\text{a4}) \Gamma
    (\text{a1}-\text{a2}+1) \Gamma (\text{a1}-\text{a3}+1) \Gamma (\text{a1}-\text{a4}+1) \Gamma (\text{b1}-\text{a1}) \Gamma (\text{b2}-\text{a1}) \Gamma (\text{b3}-\text{a1})}
    \\+\frac{\pi ^4 (-t)^{\text{a2}-1} \csc (\pi  \text{a2})
    \Gamma (\text{b1}) \Gamma (\text{b2}) \Gamma (\text{b3}) \csc (\pi  \text{a1}-\pi  \text{a2}) \csc (\pi  (\text{a2}-\text{a3})) \csc (\pi  (\text{a2}-\text{a4})) \,
    _3F_3(\text{a2}-\text{b1}+1,\text{a2}-\text{b2}+1,\text{a2}-\text{b3}+1;-\text{a1}+\text{a2}+1,\text{a2}-\text{a3}+1,\text{a2}-\text{a4}+1;-t)}{\Gamma (\text{a1}) \Gamma (\text{a2}) \Gamma (\text{a3}) \Gamma (\text{a4}) \Gamma
    (-\text{a1}+\text{a2}+1) \Gamma (\text{a2}-\text{a3}+1) \Gamma (\text{a2}-\text{a4}+1) \Gamma (\text{b1}-\text{a2}) \Gamma (\text{b2}-\text{a2}) \Gamma (\text{b3}-\text{a2})}
    \\-\frac{\pi ^4 (-t)^{\text{a3}-1} \csc (\pi  \text{a3})
    \Gamma (\text{b1}) \Gamma (\text{b2}) \Gamma (\text{b3}) \csc (\pi  \text{a1}-\pi  \text{a3}) \csc (\pi  \text{a2}-\pi  \text{a3}) \csc (\pi  (\text{a3}-\text{a4})) \,
    _3F_3(\text{a3}-\text{b1}+1,\text{a3}-\text{b2}+1,\text{a3}-\text{b3}+1;-\text{a1}+\text{a3}+1,-\text{a2}+\text{a3}+1,\text{a3}-\text{a4}+1;-t)}{\Gamma (\text{a1}) \Gamma (\text{a2}) \Gamma (\text{a3}) \Gamma (\text{a4}) \Gamma
    (-\text{a1}+\text{a3}+1) \Gamma (-\text{a2}+\text{a3}+1) \Gamma (\text{a3}-\text{a4}+1) \Gamma (\text{b1}-\text{a3}) \Gamma (\text{b2}-\text{a3}) \Gamma (\text{b3}-\text{a3})}
    \\+\frac{\pi ^4 (-t)^{\text{a4}-1} \csc (\pi  \text{a4})
    \Gamma (\text{b1}) \Gamma (\text{b2}) \Gamma (\text{b3}) \csc (\pi  \text{a1}-\pi  \text{a4}) \csc (\pi  \text{a2}-\pi  \text{a4}) \csc (\pi  \text{a3}-\pi  \text{a4}) \,
    _3F_3(\text{a4}-\text{b1}+1,\text{a4}-\text{b2}+1,\text{a4}-\text{b3}+1;-\text{a1}+\text{a4}+1,-\text{a2}+\text{a4}+1,-\text{a3}+\text{a4}+1;-t)}{\Gamma (\text{a1}) \Gamma (\text{a2}) \Gamma (\text{a3}) \Gamma (\text{a4}) \Gamma
    (-\text{a1}+\text{a4}+1) \Gamma (-\text{a2}+\text{a4}+1) \Gamma (-\text{a3}+\text{a4}+1) \Gamma (\text{b1}-\text{a4}) \Gamma (\text{b2}-\text{a4}) \Gamma (\text{b3}-\text{a4})}
\end{align}



Now consider 
$$
\phi(s) = \int_0^{\infty} x^{s - 1} \; G_{p,q}^{\,m,n} \!\left( \left. \begin{matrix} \mathbf{a_p} \\ \mathbf{b_q} \end{matrix} \; \right| \, \eta x \right) dx =
\frac{\eta^{-s} \prod_{j = 1}^{m} \Gamma (b_j + s) \prod_{j = 1}^{n} \Gamma (1 - a_j - s)} {\prod_{j = m + 1}^{q} \Gamma (1 - b_j - s) \prod_{j = n + 1}^{p} \Gamma (a_j + s)}.
$$
under the transform to get the MGF...


We can define a pair of transforms as 
$$
\mathcal{I}[f(x)](t) = \mathcal{M}^{-1}[\Gamma(s)\mathcal{M}[f(x)](1-s)](-t)
$$
$$
\mathcal{I}^{-1}[g(t)](c) = \mathcal{M}^{-1}[\frac{1}{\Gamma(1-s)}\mathcal{M}[g(-t)](1-s)](x)
$$

Particularly interesting is the action of $I^{-1} \hat{O} I f(x)$ which appears to translate polynomials into linear differential operators and back etc. We might consider how this relates to green functions... 






Consider 
$$
e^{-x} =  \frac{\text{MeijerG}\left(\{\{\},\{\}\},\left\{\left\{0,\frac{1}{2}\right\},\{\}\right\},\frac{x}{2},\frac{1}{2}\right)}{\sqrt{\pi }}
$$


$$
e^{-x} = \;_0F_0(;;-x)
$$
$$
cos(x) = \;_0F_1\left(;\frac{1}{2};-\frac{x^2}{4}\right)
$$

\end{document}