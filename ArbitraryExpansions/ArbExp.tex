\documentclass{article}

\usepackage{authblk}
\usepackage{url}
\usepackage[square,numbers]{natbib}
\usepackage{amssymb,amsmath}
\usepackage[margin=1in]{geometry}
\usepackage{graphicx}
\usepackage{setspace}
\doublespacing

%\SectionNumbersOn
%\AbstractOn

\title{Arbitrary Series Expansions}
%\author{Benedict W. J.~Irwin}

\date{\today}
\begin{document}
%\bibliographystyle{harvard}

\author[1]{Benedict W. J.~Irwin}
\affil[1]{Machine Intelligence Group, Microsoft Research, Cambridge}
\affil[]{\textit {birwin@microsoft.com}}


\maketitle

\begin{abstract}
We define a highly generalised Taylor series. This is generalised in terms of 1) the underlying calculus (additive, multiplicative, ...) and 2) the underlying basis function for the expansion.
\end{abstract}

\section{Main}
Define the generalised sum, product and derivative with respect to forward and backward functions $\phi$ and $\phi^{-1}$ respectively, who are complementary inverses.

\begin{align}
a \oplus b = \phi^{-1}( \phi(a) + \phi(b) ) \\
a \otimes b = \phi^{-1}( \phi(a) \times \phi(b) ) \\
D_{\phi,x} f(x) = \phi^{-1}( D_x \phi(f(x)) )
\end{align}

These have extensions for $\Sigma$ and $\Pi$ big operators.

One can show a $\phi$-generalised Taylor series for a function is
$$
f(x+x_0) = \bigoplus_{k=0}^{\infty} \left[D^{k}_{\phi, x}f(x) \right]_{x=x_0} \otimes \phi^{-1}\left( \frac{(x-x_0)^k}{k!}\right)
$$

However, this expansion is limited to a power series, using a monomial basis... 

D.Widden showed an extension to the Taylor series as
$$
f(x) = \sum_{k=0}^\infty \left[L_k f(x)\right]_{x=t} g_k(x,t)
$$
where defining an input basis function $\varphi_k(x)$, we have
\begin{align}
g_k(x,t) = \frac{\int_t^x \prod_{l=0}^k \varphi_l(s)g_{k-1}(x,s)\;ds}{\prod_{l=0}^k \varphi_l(t)} \\
L_k f(x) = \frac{W[\{g_0(x,t),\cdots,g_{k-1}(x,t),f(x)\}]}{W[\{g_0(x,t),\cdots,g_{k-1}(x,t)\}]}
\end{align}
where $W$ is the Wronskian.

\section{Next}
We conjecture it is possible to simply combine these two extensions, primarily by modifying both the definition of the Wronskian (or derivative) and the integral to that of the alternative calculus. We denote this final expansion as

$$
f(x) = \bigoplus_{k=0}^{\infty} \left[L^*_k f(x)\right]_{x=t} \otimes g^*_k(x,t)
$$
where
\begin{align}
g^*_k(x,t) = \frac{\int_t^x \prod_{l=0}^k \varphi_l(s)g_{k-1}(x,s)\;ds}{\prod_{l=0}^k \varphi_l(t)} \\
L^*_k f(x) = \frac{W^*[\{g^*_0(x,t),\cdots,g^*_{k-1}(x,t),f(x)\}]}{W^*[\{g^*_0(x,t),\cdots,g^*_{k-1}(x,t)\}]}
\end{align}
we now use a modified Wronskian $W^*$ which is defined as 
$$
W^*_{n}[\{f_0(x), \cdots, f_n(x)\}] = \begin{vmatrix}
f_0(x) & \cdots & f_n(x) \\
f^*_0(x) & \cdots & f^*_n(x) \\
\vdots & & \vdots \\
f^{*(n)}_0(x) & \cdots & f^{*(n)}_n(x)
\end{vmatrix}
$$

we also have to consider changing 1) the definition of the integral itself, 2) consider using the modified product in place of the product for the functions $\varphi_k$ in the definition of $g_k(x,t)$.

We could even end up having to define a modified determinant, which would be most unfortunate.

For regular calculus $\phi_k(x) = 1$ and we have 
$$
W_3(x,t) = \det
\left(
                   \begin{array}{cccc}
                    1 & x-t & \frac{1}{2} (t-x)^2 & -\frac{1}{6} (t-x)^3 \\
                    0 & 1 & x-t & \frac{1}{2} (t-x)^2 \\
                    0 & 0 & 1 & x-t \\
                    0 & 0 & 0 & 1 \\
                   \end{array}
                   \right)
$$
whereas 
$$
\left(
                   \begin{array}{cccc}
                    1 & -\sin ^{-1}(\sin (t-x)) & \sin ^{-1}\left(\sin \left(\frac{1}{2} (t-x)^2\right)\right) & -\sin ^{-1}\left(\sin \left(\frac{1}{6} (t-x)^3\right)\right) \\
                    0 & \sin ^{-1}(\cos (t-x)) & \sin ^{-1}\left((x-t) \cos \left(\frac{1}{2} (t-x)^2\right)\right) & \csc ^{-1}\left(\frac{2 \sec \left(\frac{1}{6} (t-x)^3\right)}{(t-x)^2}\right) \\
                    0 & \sin ^{-1}(\sin (t-x)) & \sin ^{-1}\left(\cos \left(\frac{1}{2} (t-x)^2\right)-(t-x)^2 \sin \left(\frac{1}{2} (t-x)^2\right)\right) & -\sin ^{-1}\left((t-x) \cos \left(\frac{1}{6}
                      (t-x)^3\right)-\frac{1}{4} (t-x)^4 \sin \left(\frac{1}{6} (t-x)^3\right)\right) \\
                    0 & -\sin ^{-1}(\cos (t-x)) & -\sin ^{-1}\left(3 (x-t) \sin \left(\frac{1}{2} (t-x)^2\right)+(x-t)^3 \cos \left(\frac{1}{2} (t-x)^2\right)\right) & -\sin ^{-1}\left(\frac{3}{2} (t-x)^3
                      \sin \left(\frac{1}{6} (t-x)^3\right)+\frac{1}{8} \left((t-x)^6-8\right) \cos \left(\frac{1}{6} (t-x)^3\right)\right) \\
                   \end{array}
                   \right)
$$


\section{Arbitrary Convolutions}
We have additive $z=x+y$, convolution of distributions
$$
p(z) = \int_{-\infty}^\infty p(x)p(z-x)\;dx
$$
and multiplicative $z = x y$
$$
p(z) = \int_{-\infty}^\infty p(x)p\left(\frac{z}{x}\right)\; \frac{dx}{|x|}
$$

here
$$
\frac{1}{|x|} = \int_{-\infty}^\infty \delta(z - xy) \; dy
$$
to generalise this we use
$$
p(z) = \int_{-\infty}^\infty p(x)p\left(\phi^{-1}(\phi(z)) \ominus \phi(x)\right)\; \frac{dx}{\mu(x,z)}
$$
where 
$$
\mu(x,z) = \int_{-\infty}^\infty \delta(y - \phi^{-1}(\phi(z)) \ominus \phi(x))\; dy
$$
using the identity
$$
\delta(g(x)) = \sum_{\rho_g} \frac{\delta(x - \rho)}{|g'(\rho)|}
$$

\section{Arbitrary Number Theory/Divisors}
We can imagine redefining
$$
\sum_{d|n} \mu(n)
$$
with the idea that 
$$
\phi^{-1}(6) \oslash \phi^{-1}(2) = \phi^{-1}(3)
$$


\section{Attempt to systematically generalise Mellin transform}

Gamma as an infinite product 
$$
\Gamma(z) = \frac{1}{z} \prod_{n=1}^\infty \frac{(1 + \frac{1}{n})^z}{1 + \frac{z}{n}}
$$
but also 
$$
\Gamma(z) = \int_0^\infty x^{z-1} e^{-x}\; dx
$$

Extension of factorial
$$
\prod \to 1 \cdot 2 \cdot 3 \cdot 4 \to \exp\left( \log(1) + \log(2) + \log(3) + \log(4)  \right) \to \bigoplus_\phi
$$

$$
\Gamma(n+1) \to n! = \prod_{k=1}^n k = \bigoplus_{\exp \to \log | k = 1}^n k = n!_{\exp \to \log} \to n!_{\phi}
$$
Then the 'factorials' for the identity transform are the triangular numbers...

For 
$$
n!_{\tanh^{-1} \to \tanh}
$$
we have an expression in terms of ArcTanh of QPolyGamma functions.
$$
 \left\{1,-\tanh ^{-1}\left(\frac{2-2 e^6}{1+e^2+e^4+e^6}\right),-\tanh ^{-1}\left(\frac{3-2 e^2+3 e^4-3 e^6+2 e^8-3 e^{10}}{1+e^4+e^6+e^{10}}\right),-\tanh ^{-1}\left(\psi _e^{(0)}\left(1-\frac{i \pi }{2}\right)-\psi _e^{(0)}\left(5-\frac{i \pi
    }{2}\right)+\psi _e^{(0)}\left(1+\frac{i \pi }{2}\right)-\psi _e^{(0)}\left(5+\frac{i \pi }{2}\right)+4\right)\right\}
$$
we can imagine these rational polynomial in $e$ expressions have a nice form somewhere...

We seem to have 
$$
n!_{x^2 \to \sqrt{x}} = H_{n,-\frac{1}{2}}^2
$$
an strong link to zeta functions. And for the first root
$$
n!_{x^k \to x^{1/k}} = H_{n,-\frac{1}{k}}^k
$$
of which when $k \to 1$ we get the triangular numbers
$$
n!_{Id \to Id} = \frac{n(n+1)}{2}
$$
so we can look for the integral definition and infinite product definition of generalised harmonic numbers for inspiration
$$
H_n = \int_0^1 \frac{1 - x^n}{1-x} \; dx
$$
we want 
$$
H_{n,m} = \sum_{k=1}^n \frac{1}{k^m}
$$
For positive integers $r$
$$
H_z^r = \frac{(-1)^{r-1}}{(r-1)!}\int_0^1 \frac{(t^z-1)\log^{r-1}(t)}{t-1} \; dt
$$
which looks linked to the derivative of the Mellin transform in order to accumulate the powers of $\log$ in the integrand.



\subsection{Triangular Numbers Infinite product}
So consider we try to edit the infinite product for gamma and 'undo' the $\exp \to \log$ aspect of it.

$$
\Gamma(z) = \exp\left( \log\frac{1}{z} + \sum_{n=1}^\infty \log \frac{(1 + \frac{1}{n})^z}{1 + \frac{z}{n}} \right)
$$
try
$$
\Delta_?(z) = \frac{1}{z} + \sum_{n=1}^\infty \frac{(1 + \frac{1}{n})^z}{1 + \frac{z}{n}}
$$
seems to diverge... do we need a regularised sum? Cesaro/Ramanujan? It might be that the exponentiation is actually an example of $\phi^{-1}$ i.e.e $\exp$ being triggered...
$$
(1+ \frac{1}{n})^z = \exp( z \log( 1 + \frac{1}{n}))
$$

We also have the Wierstrauss version of the infinite product formula
$$
\Gamma(z) =  \frac{e^{- \gamma z}}{z}\prod_{n=1}^\infty (1 + \frac{z}{n})^{-1} e^{z/n}
$$


\subsection{Simple Polynomials}
Consider
$P(x) = a x^2 + b x + c$
we can get the inverse as a root.

We end up with the factorial like 
$$
\frac{4 \sqrt{a} b \left(\zeta \left(-\frac{1}{2},\frac{b^2}{4 a}-c+1\right)-\zeta \left(-\frac{1}{2},\frac{b^2}{4 a}-c+n+1\right)\right)+\left(2 \sqrt{a} \left(\zeta \left(-\frac{1}{2},\frac{b^2}{4 a}-c+n+1\right)-\zeta \left(-\frac{1}{2},\frac{b^2}{4
    a}-c+1\right)\right)+b n\right)^2+4 a c-2 b^2 n}{4 a}
$$
with first 3 terms like 
$$
     \left\{1,\frac{-\sqrt{a} b \left(\sqrt{\frac{b^2}{a}-4 c+4}+\sqrt{\frac{b^2}{a}-4 c+8}\right)+a \left(\sqrt{\frac{b^2}{a}-4 c+4} \sqrt{\frac{b^2}{a}-4 c+8}-2 c+6\right)+b^2}{2 a},\frac{\left(\sqrt{a} \left(\zeta \left(-\frac{1}{2},\frac{b^2}{4
    a}-c+1\right)-\zeta \left(-\frac{1}{2},\frac{b^2}{4 a}-c+4\right)\right)-2 b\right) \left(\zeta \left(-\frac{1}{2},\frac{b^2}{4 a}-c+1\right)-\zeta \left(-\frac{1}{2},\frac{b^2}{4 a}-c+4\right)\right)}{\sqrt{a}}+\frac{3 b^2}{4 a}+c\right\}
$$
which is nice that 1 is a first term. We should be able to get this to collapse into the $x^2, \sqrt{x}$ case by setting the coefficients correctly. This appears to work, but arrives in a new angle in terms of zeta functions.

\subsection{General Series Reversion?}
We can the consider the case where we have a forward function that has an analytic series expansion and the backward function who has coefficients determinable by a series reversion of the forward function. 

Of course the generalised form of the inverse coefficients is complicated (i.e. Morse\&Fesbach), but in principle this sets a route for calculating the generalised factorials of arbitrary series expansions. 

For some functions (i.e. fifth order polynomial) we would then end up with a hypergeometric function as the root. This brings us closer to understanding the origin of the hypergeometric functions that arrive in the moments of certain functions? 

\subsection{Circle}
A circular transform might be a common thing, with $\sqrt{x} \sqrt{1-x}$ we get inverse function 
$$
\frac{1}{2} \pm \frac{\sqrt{1 - 4 x^2
}}{2}
$$
The 'factorials' look like nested surd/radicals.


For the reverse of the product calculus we get a simple almost linear thing
$$
n!_{\log \to \exp} \sim n + 1 - \log(e-1)
$$
$$
n!_{W(x) \to x\exp(x)} \sim n + 1 - \log(e-1)
$$

A beautiful result is 
$$
n!_{1-\frac{1}{x} \to \frac{1}{1-x}} = \frac{1}{1 - n + H_n}
$$
$$
n!_{\frac{1}{1+x} \to \frac{1}{x}-1 } = \frac{1}{1 -                                                                                   n + H_n}
$$
and probably more fundamental is 
$$
n!_{\frac{1}{x} \to \frac{1}{x}} = \frac{1}{H_n}
$$

$$
n!_{1-e^x \to \log(1-x)} = 1 - \frac{1}{\Gamma(1-n)}
$$

\section{Flexible Transform}
Consider $\lambda \in [0,1]$
$$
\phi(x) = \lambda x + (1-\lambda) \log(x)
$$
$$
\phi^{-1}(x) = \lambda x + (1-\lambda) \exp(x)
$$
so when lambda is $1$ we have addition, and $\lambda=0$ is multiplication.

but also the true inverses like
$$
\phi_{\log}^{-1} = -\frac{(\lambda -1) W\left(-\frac{\lambda  \left(e^x\right)^{-\frac{1}{\lambda -1}}}{\lambda -1}\right)}{\lambda }
$$

For this we have 
$$
\lim_{\lambda \to 0} \phi_{\log}^{-1} = e^x
$$
and we would expect 



$$
\phi_{\exp}^{-1} = \frac{x}{\lambda }-W\left(-\frac{(\lambda -1) e^{\frac{x}{\lambda }}}{\lambda }\right)
$$
For this we have $x$ when $\lambda = 1$, and we would expect $\log(x)$, in the other limit.


This is very cool. For example, we can ask questions like, that value of lambda interpolates $1+1 = 2$ and $1*1 = 1$, for something like $1 \oplus 1 = \sqrt{2}$

For $\lambda = 1/2$ we have $1 \oplus 1 =  \frac{1}{2} \left(W\left(e^2\right)-2\right) W\left(e^2\right)+2 \approx 1.65521$

There is some cool stuff going on here... But will need to consider carefully what is actually going on, and whether it is useful outside of the two points $0,1$... Is it a fuzzy and/or for probabilities for example? Can we use it inside an integral transform or differential equation to somehow interpolate between different solutions i.e. Gamma function and triangular numbers? 

I.e. if we define a generalised recurrence relation that 
$$
f(x + 1) = \phi^{-1}( \phi(f(x)) + \phi(x)) 
$$
this can interpolate between $f(x+1) = x f(x)$ i.e. gamma function and $f(x+1) = f(x) + x$ i.e. triangular number.... 

$$
\lambda  \left(W\left(-\frac{(\lambda -1) e^{\frac{f(x)}{\lambda }}}{\lambda }\right)+W\left(-\frac{(\lambda -1) e^{\frac{x}{\lambda }}}{\lambda }\right) \left(\frac{\lambda  W\left(-\frac{(\lambda -1) e^{\frac{f(x)}{\lambda }}}{\lambda
    }\right)}{\lambda -1}+1\right)\right)+f(x+1)=f(x)+x
$$
    
We would also have an interpolation between the derivative and the logarithmic derivative?

\section{Alternative Gamma}
There are other integrals that look a bit more like an $\exp \to \log$ integral...

$$
\Gamma(z) = \exp\left(\int_0^1 \frac{x^z-z(x-1)-1}{(x-1)\log(x)} \; dx\right)
$$
$$
\Gamma(z) = e^{-\gamma z}\exp\left(\int_0^1 \frac{x^z-\log(x^z)-1}{(x-1)\log(x)} \; dx\right)
$$

we want this to be in a 'product integral' type form
$$
I = \exp\left(\int_0^1 \log f(x) \; dx\right)
$$

Of course one definition for triangular numbers is simply 
$$
\Delta(n) = \int_0^1 (n+n^2)x \; dx
$$



\bibliography{bibliography}{}
\bibliographystyle{unsrt}


\end{document}