\documentclass{article}

\usepackage{authblk}
\usepackage{url}
\usepackage[square,numbers]{natbib}
\usepackage{color,amssymb,amsmath}
\usepackage{graphicx}
\usepackage[margin=1in]{geometry}
%\usepackage{graphicx}
%\SectionNumbersOn
%\AbstractOn

\title{Moebius Graph Walk}
%\author{Benedict W. J.~Irwin}


\date{\today}
\begin{document}

%\email{ben.irwin@optibrium.com}
%\affiliation{Optibrium, F5-6 Blenheim House, Cambridge Innovation Park, Denny End Road, Cambridge, CB25 9PB, United Kingdom}
%\alsoaffiliation{Theory of Condensed Matter, Cavendish Laboratories, University of Cambridge, Cambridge, United Kingdom}

\author[1,2]{Benedict W. J.~Irwin}
\affil[1]{Theory of Condensed Matter, Cavendish Laboratories, University of Cambridge, Cambridge, United Kingdom}
\affil[2]{Optibrium, F5-6 Blenheim House, Cambridge Innovation Park, Denny End Road, Cambridge, CB25 9PB, United Kingdom}
\affil[ ]{\textit {ben.irwin@optibrium.com}}


\maketitle

\begin{abstract}
We find a graph representation of the integers.
\end{abstract}

\section{Rules}
The method for generating the graph representation is governed by the following rules:
\begin{itemize}
\item Each node (an integer $n$) must have a 'flux' equal to $\mu(n)$. The flux is the sum of inward pointing edges minus the sum of outward pointing edges.
\item Each node has $2-\mu(n)$ edges in total.
\end{itemize}
the graph may be procedurally generated and new labels $n$ are assigned to nodes upon creation of an edge to balance the flux.



\section{Graph Breaks}
The second rule was brought in for a consistent way to resolve breaks in the graph. These were found to occur when Mertens function $M(n)$ equals zero for a node $n$.

\section{Backbone Integers}
These will either be prime or numbers that are not square free, or ...

\section{Peripheral Integers}

\section{Twin Primes Share an Edge}
Twin primes should always be connected (unless.... 

\section{Gaussian Primes}
A318608		Moebius function mu(n) defined for the Gaussian integers.
This appears to form a chain? With no branches???


\section{Dirichlet Convolution}
we have $\mu * 1 = \varepsilon$ with $\varepsilon(1)=1$ and otherwise 0. $\varepsilon$ generates a chain where the initial $1$ creates an impulse and the following numbers propagate this endlessly in the simplest forward directional chain possible. The function $1$, as a DGF is $\zeta(s)$ which creates a cascade where $1$ feeds from $2$ which branches from $3,4$ which branch into $5,6,7,8$ and so on, because each node wants to maintain a flux of $1$. Note the direction of the arrows.

The Dirichlet convolution is defined over divisors, and thus divisors play a role in these graphs. The convolution of two graphs will depend on the divisors and the values of the function at each node.

Adding and cancelling the edges of a graph does not convolve them, it appears to give the subtraction of the functions [check this?], for example stacking the edges of $1/\zeta(s)$ and $\zeta(s)$ appears to give a graph for $\mu(n)-1$. [My zeta was the wrong way round, it gives the sum of the two as expected].

We might imagine a kind of divisor oriented matrix multiplication of the adjacency matrices. 

The degree will be sum of the column minus sum of the row in terms of 'flux'.
$$
\mathrm{deg}_n(A) = \sum_{k=1}^N (A_{kn}-A_{nk})
$$
the idea is that for matrices $A$ $B$ and $C$ we have 
$$
a(n) = \mathrm{deg}_n(A)
$$
$$
b(n) = \mathrm{deg}_n(B)
$$
$$
c(n) = \mathrm{deg}_n(C)
$$
and 
$$
c(n) = (a * b)(n)
$$
we are searching for the operation $\otimes$ such that
$$
(a * b)(n) = \mathrm{deg}_n(A \otimes B) = \sum_{k=1}^N ([A \otimes B]_{kn}-[A \otimes B]_{nk})
$$
alternatively
$$
\sum_{d|n}a(d)b\left(\frac{n}{d}\right) = \sum_{k=1}^N ([A \otimes B]_{kn}-[A \otimes B]_{nk})
$$
and
$$
\sum_{d|n}b(d)a\left(\frac{n}{d}\right) = \sum_{k=1}^N ([A \otimes B]_{kn}-[A \otimes B]_{nk})
$$
for example
$$
\sum_{d|n}\mathrm{deg}_d(A)\mathrm{deg}_{n/d}(B) = \sum_{k=1}^N ([A \otimes B]_{kn}-[A \otimes B]_{nk})
$$
$$
\sum_{d|n}\left(\sum_{k=1}^N (A_{kd}-A_{dk})\right)\left(\sum_{l=1}^N (B_{l,n/d}-B_{n/d,l})\right) = \sum_{k=1}^N ([A \otimes B]_{kn}-[A \otimes B]_{nk})
$$
$$
\sum_{d|n}\sum_{k=1}^N \sum_{l=1}^N  (A_{kd}-A_{dk})(B_{l,n/d}-B_{n/d,l}) = \sum_{k=1}^N ([A \otimes B]_{kn}-[A \otimes B]_{nk})
$$
$$
\sum_{k=1}^N \sum_{d|n} \sum_{l=1}^N  (A_{kd}-A_{dk})(B_{l,n/d}-B_{n/d,l}) = \sum_{k=1}^N ([A \otimes B]_{kn}-[A \otimes B]_{nk})
$$
thus at least contender for this property (for a given $k$) is
$$
\sum_{d|n} \sum_{l=1}^N  (A_{kd}-A_{dk})(B_{l,n/d}-B_{n/d,l}) = [A \otimes B]_{kn}-[A \otimes B]_{nk}
$$
$$
\sum_{d|n} \sum_{l=1}^N  (A_{kd}B_{l,n/d}-A_{dk}B_{l,n/d} + A_{dk}B_{n/d,l}-A_{kd}B_{n/d,l}) = [A \otimes B]_{kn}-[A \otimes B]_{nk}
$$
the divisor symmetry being $d \leftrightarrow n/d$, the Dirichlet convolution should also commute.
$$
\sum_{d|n} \sum_{l=1}^N  (A_{kd}B_{l,n/d} -A_{kd}B_{n/d,l} ) - \sum_{d|n} \sum_{l=1}^N (A_{dk}B_{l,n/d} - A_{dk}B_{n/d,l}) = [A \otimes B]_{kn}-[A \otimes B]_{nk}
$$
this has nice properties in terms of the negated transpose of $B$.

\section{Divisor Sigma}
The graph for divisor sigma (0) is exquisite, it seems to have layers of Bell numbers and a prime for every node?
We are likely getting the concept of flux wrong as well. The number in the Dirichlet sequence is the number of inward arrows going to that node... 

$1 \leftarrow 2 \leftarrow (3,4) \leftarrow (5-9) \leftarrow $

Seems to relate to $A054519 + 1$

The above is false, it is the prime gaps sequence/graph that gives the correct allocation of a prime to each node.


In this graph, a prime has inputs from nodes $p_n$ to $p_{n+1}-1$.



\bibliography{bibliography}{}
\bibliographystyle{plain}


\end{document}